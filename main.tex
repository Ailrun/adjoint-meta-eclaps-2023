\documentclass[letterpaper,12pt,aspectratio=169,show notes,dvipsnames]{beamer}

\setbeameroption{show notes on second screen=right}

\usepackage{amsmath}
\usepackage{amssymb}
\usepackage{amsthm}
\usepackage{attrib}
\usepackage{bussproofs}
\usepackage{charter}
\usepackage{fontspec}
\usepackage{listings}
\usepackage{microtype}
\usepackage{nicefrac}
\usepackage{tikz}
\usetikzlibrary{calc,shapes.callouts}
\usepackage{upquote}
\usepackage{wasysym}

\setmainfont[Mapping=tex-text]{Noto Sans}
\newfontfamily\NotoSans[Mapping=tex-text]{Noto Sans}
\newfontfamily\NotoSerif[Mapping=tex-text]{Noto Serif}
\setmonofont{JuliaMono}

\definecolor{darkgreygreen}{RGB}{32, 39, 41}
\definecolor{grey}{rgb}{0.7, 0.7, 0.7}
\setbeamercolor{structure}{fg=darkgreygreen}

\beamertemplatenavigationsymbolsempty
\setbeamercolor{footline}{fg=structure}
\setbeamerfont{footline}{series=\bfseries,size=\scriptsize}
\addtobeamertemplate{navigation symbols}{}{%
  \usebeamerfont{footline}%
  \usebeamercolor[fg]{footline}%
  \hspace{1em}%
  \insertframenumber/\inserttotalframenumber
}

\mode<presentation>
{
  \useinnertheme{rectangles}
  \useoutertheme{default}
  \usecolortheme{whale}
  % \setbeamerfont{block title}{size=\large}
  \setbeamertemplate{itemize item}{\color{grey}$\blacktriangleright$}
  \setbeamertemplate{itemize subitem}{\color{grey}\scriptsize{$\blacktriangleright$}}
  % \setbeamercolor{titlelike}{parent=structure,bg=white!85!MSUgreen}
  % \setbeamercovered{transparent}
}

\setbeamerfont{title}{family=\NotoSerif}
\setbeamerfont{author}{family=\NotoSerif}
\setbeamerfont{frametitle}{family=\NotoSerif}
\makeatletter
\setbeamertemplate{title page}{%
  \vbox{}
  \vfill
  \begingroup
  \hfill
  \begin{beamercolorbox}[wd=0.6\paperwidth,sep=10pt,center]{title}
    \usebeamerfont{title}\inserttitle\par%
    \ifx\insertsubtitle\@empty%
    \else%
      \vskip0.25em%
      {\usebeamerfont{subtitle}\usebeamercolor[fg]{subtitle}\insertsubtitle\par}%
    \fi%
  \end{beamercolorbox}%
  \hfill
  \hfill
  \vskip2em\par
  \hfill
  \begin{beamercolorbox}[wd=0.6\paperwidth,sep=8pt,center]{author}
    \usebeamerfont{author}\insertauthor
  \end{beamercolorbox}
  \hfill
  \hfill
  \par
  \hfill
  \begin{beamercolorbox}[wd=0.6\paperwidth,sep=8pt,center]{institute}
    \usebeamerfont{institute}\insertinstitute
  \end{beamercolorbox}
  \hfill
  \hfill
  % \par
  % \hfill
  % \begin{beamercolorbox}[sep=8pt,center]{date}
  %   \usebeamerfont{date}\insertdate
  % \end{beamercolorbox}\vskip0.5em
  % \hfill
  % \hfill
  \endgroup
  \vfill
}
\makeatother

\setbeamertemplate{section page}{
  \begin{beamercolorbox}[sep=12pt,center]{section title}
    \usebeamerfont{section title}\insertsection\par
  \end{beamercolorbox}
}
\setbeamertemplate{frametitle}[default][left,sep=0.5cm]

\AtBeginDocument{%
  \title{Multi-modal Programming with Resource Guarantees}
  \date{}
  \author[C. Jang]{Junyoung/``Clare'' Jang}
  \institute{McGill University}
}

\pgfkeys{/tikz/fulloverlay/.code={\pgfkeysalso{overlay, remember picture, shift=(current page.south west), x=(current page.south east), y=(current page.north west)}}}
\tikzset{
  every node/.style={font=\NotoSans},
  invisible/.style={opacity=0,text opacity=0},
  visible on/.style={alt={#1{}{invisible}}},
  alt/.code args={<#1>#2#3}{%
    \alt<#1>{\pgfkeysalso{#2}}{\pgfkeysalso{#3}} % \pgfkeysalso doesn't change the path
  },
}

\lstdefinelanguage{Adjoint}%
{%
  keywords=%
  {%
    box, %
    lift, unlift,liftp,unliftp,%
    return,llet,let,in,%
    returnp,%
    nat,suc,%
    array,%
    fun,%
    if,then,else,%
    match,with,private%
  },%
  morecomment=[l]{--},%
  moredelim=[is][\color{orange}]{o<<}{>>o},%
  moredelim=[is][\color{violet}]{v<<}{>>v},%
  moredelim=[is][\color{NavyBlue}]{nb<<}{>>nb},%
  moredelim=[is][\color{WildStrawberry}]{ws<<}{>>ws},%
  sensitive,
}
\lstset{%
  basicstyle=\ttfamily\footnotesize,%
  keywordstyle=\bfseries,%
  literate=%
  {->}{{\(\rightarrow\)}}2 %
  {-o}{{\(\multimap\)}}2 %
  {^|}{{\(\uparrow\)}}2 %
  {v|}{{\(\downarrow\)}}2 %
  {b[]b}{{\(\Box\)}}2 % box
  {\\alpha}{{\(\alpha\)}}1 %
  {\\beta}{{\(\beta\)}}1 %
  {\\tau}{{\(\tau\)}}1 %
  {\\sigma}{{\(\sigma\)}}1 %
  {\\rho}{{\(\rho\)}}1 %
  ,%
  keepspaces=true,%
  % numbers=left,%
  mathescape=true,%
  xleftmargin=0.05in,%
  resetmargins=true,%
  belowskip=0.05in,%
  aboveskip=0.05in,%
}
\lstloadlanguages{Adjoint}
\lstset{language=Adjoint}

\colorlet{metaprogram}{violet}
\colorlet{linearity}{blue}
\colorlet{privacy}{orange}
\begin{document}
\begin{frame}
  \titlepage
\end{frame}

\begin{frame}[fragile]{Useful Modalities for Various Applications}
  \begin{tikzpicture}[fulloverlay]
	\node[visible on=<2->,metaprogram] (mp) at (0.25,0.65) {\bfseries Metaprogramming};

    \node[visible on=<3->,anchor=base,overlay,rectangle callout,
    callout relative pointer={(0.1cm,0.3cm)},text width=10em,
    draw=metaprogram,color=metaprogram,align=center]
    at ($(mp.south)-(0.1cm,\ht0+0.6cm)$) {How to characterize code fragments?};

	\node[visible on=<4->,linearity] (ra) at (0.6,0.5) {\bfseries Resource Availability};

    \node[visible on=<5->,anchor=base,overlay,rectangle callout,
    callout relative pointer={(-0.1cm,-0.3cm)},text width=12em,
    draw=linearity,color=linearity,align=center]
    at ($(ra.north)+(0.1cm,\ht0+0.9cm)$) {How many resources are available?};

	\node[visible on=<6->,privacy] (rp) at (0.4,0.3) {\bfseries Resource Privacy};

    \node[visible on=<7->,anchor=west,overlay,rectangle callout,
    callout relative pointer={(-0.3cm,-0.02cm)},text width=10em,
    draw=privacy,color=privacy,align=center]
    at ($(rp.east)+(\ht0+0.3cm,0.2cm)$) {Is resource\\access permitted?};
  \end{tikzpicture}
\end{frame}

\begin{frame}{Modalities in Essence}
  \begin{tikzpicture}[fulloverlay]
	\node[metaprogram!60] (mp) at (0.25,0.65) {\bfseries Metaprogramming};
	\node[visible on=<2->,metaprogram,text width=14em,align=center] at ($(mp.south)-(-1cm,\ht0+0.3cm)$)
    {\bfseries \alt<5->{{\color{white}Code/Program}}{{Code/Program}} Distinction\\with Intensionality};

	\node[linearity!60] (ra) at (0.6,0.5) {\bfseries Resource Availability};
	\node[visible on=<3->,linearity,text width=17em,align=center] at ($(ra.south)-(-1.3cm,\ht0+0.3cm)$)
    {\bfseries \alt<5->{{\color{white}Linear/Unrestricted}}{{Linear/Unrestricted}} Distinction\\with Structural Rules};

	\node[privacy!60] (rp) at (0.4,0.3) {\bfseries Resource Privacy};
	\node[visible on=<4->,privacy,text width=14em,align=center] at ($(rp.south)-(-1.1cm,\ht0+0.1cm)$)
    {\bfseries \alt<5->{{\color{white}Private/Public}}{{Private/Public}} Distinction};

	\draw[visible on=<6->,rounded corners=0.5cm,draw=darkgreygreen] (0.05,0.75) rectangle (0.95,0.2);
	\node[visible on=<6->,color=darkgreygreen,fill=white] at (0.5,0.75) {Single Uniform Framework};
  \end{tikzpicture}
\end{frame}

\begin{frame}{Modalities for Distinction}
  \begin{tikzpicture}[fulloverlay]
    \node[visible on=<-1>,anchor=west] at (0.05,0.75) {For any two sectors \(m_1\) and \(m_2\),};
    \node[visible on=<2>,anchor=west] at (0.05,0.75) {For \color{metaprogram}{\bfseries metaprogramming}};
    \node[visible on=<3>,anchor=west] at (0.05,0.75) {For \color{privacy}{\bfseries resource privacy}};

    \draw
    (0.15,0.6) rectangle (0.45,0.45)
    (0.15,0.45) rectangle (0.45,0.3);
    \node[visible on=<-1>] at (0.3,0.525) {\(m_2\)};
    \node[visible on=<-1>] at (0.3,0.375) {\(m_1\)};
    \node[visible on=<2>,metaprogram] at (0.3,0.525) {\(c\) : Code};
    \node[visible on=<2>,metaprogram] at (0.3,0.375) {\(p\) : Program};
    \node[visible on=<3>,privacy] at (0.3,0.525) {\(p\) : \parbox{8em}{\centering Program with Public Resource}};
    \node[visible on=<3>,privacy] at (0.3,0.375) {\(s\) : Secure Resource};

    \node[visible on=<-1>] at (0.7,0.525) {\(\uparrow^{m_2}_{m_1}\) : \parbox{10em}{\centering A modality ``going up'' from \(m_1\) to \(m_2\)}};
    \node[visible on=<-1>] at (0.7,0.375) {\(\downarrow^{m_2}_{m_1}\) : \parbox{10em}{\centering A modality ``going down'' from \(m_2\) to \(m_1\)}};
    \node[visible on=<2>] at (0.7,0.525) {\(\uparrow^{c}_{p}\) : \parbox{10em}{\centering A modality ``going up'' from \(p\) to \(c\)}};
    \node[visible on=<2>] at (0.7,0.375) {\(\downarrow^{c}_{p}\) : \parbox{10em}{\centering A modality ``going down'' from \(c\) to \(p\)}};
    \node[visible on=<3>] at (0.7,0.525) {\(\uparrow^{p}_{s}\) : \parbox{10em}{\centering A modality ``going up'' from \(s\) to \(p\)}};
    \node[visible on=<3>] at (0.7,0.375) {\(\downarrow^{p}_{s}\) : \parbox{10em}{\centering A modality ``going down'' from \(p\) to \(s\)}};
  \end{tikzpicture}
\end{frame}

\begin{frame}{Going-up Modality}
  \begin{tikzpicture}[fulloverlay]
    \node at (0.5,0.6) {\(\mathsf{lift}^{m_2}_{m_1} (M)\) : \parbox{19em}{\centering Lift an expression \(M\) in \(m_1\) to its AST in \(m_2\)\\{}It is of type \(\uparrow^{m_2}_{m_1} A\) in \(m_2\) if \(M : A\) in \(m_1\)}};
    \node at (0.5,0.4) {\(\mathsf{unlift}^{m_2}_{m_1} (M)\) : Execute/splice-in an AST \(M\)};
  \end{tikzpicture}
\end{frame}

\begin{frame}{Going-down Modality}
  \begin{tikzpicture}[fulloverlay]
    \node at (0.5,0.6) {\(\mathsf{return}^{m_2}_{m_1} (M)\) : \parbox{22em}{\centering Store an expression \(M\) in \(m_2\) for future use in \(m_2\)\\{}It is of type \(\downarrow^{m_2}_{m_1} A\) in \(m_1\) if \(M : A\) in \(m_2\)}};
    \node at (0.5,0.4) {\(\mathsf{let}\ \mathsf{return}^{m_2}_{m_1} x = M\ \mathsf{in}\ N\) : \parbox{22em}{\centering Load the stored value pointed by \(M\) into \(x\)\\and continue with \(N\)}};
  \end{tikzpicture}
\end{frame}

\begin{frame}[fragile]{Metaprogramming Example}
  \begin{tikzpicture}[fulloverlay]
    \node[text width=0.58\paperwidth,visible on=<1>] at (0.31,0.6) {
      \begin{lstlisting}
pow : nat -> v|$^c_p$^|$^c_p$(nat -> nat)
pow 0       = return$^c_p$ (lift$^c_p$ (fun x -> 1))
pow (suc n) =
  let return$^c_p$ C = pow n in
  return$^c_p$ (lift$^c_p$ (fun x -> x * (unlift$^c_p$ (C)) x))
      \end{lstlisting}
    };

    \node[text width=0.58\paperwidth,visible on=<2>] at (0.31,0.6) {
      \begin{lstlisting}
pow : nat -> v|$^c_p$^|$^c_p$(nat -> nat)
pow 0       = nb<<return$\color{NavyBlue}^c_p$ (lift$\color{NavyBlue}^c_p$ (fun x -> 1))>>nb
pow (suc n) =
  let return$^c_p$ C = pow n in
  return$^c_p$ (lift$^c_p$ (fun x -> x * (unlift$^c_p$ (C)) x))
      \end{lstlisting}
    };
    \draw[NavyBlue,visible on=<2>] (0.415,0.66) -- (0.55,0.66);

    \node[text width=0.58\paperwidth,visible on=<3>] at (0.31,0.6) {
      \begin{lstlisting}
pow : nat -> v|$^c_p$^|$^c_p$(nat -> nat)
pow 0       = return$^c_p$ (lift$^c_p$ (fun x -> 1))
pow (suc n) =
  ws<<let return$\color{WildStrawberry}^c_p$ C = pow n in>>ws
  return$^c_p$ (lift$^c_p$ (fun x -> x * (unlift$^c_p$ (C)) x))
      \end{lstlisting}
    };

    \node[text width=0.58\paperwidth,visible on=<4>] at (0.31,0.6) {
      \begin{lstlisting}
pow : nat -> v|$^c_p$^|$^c_p$(nat -> nat)
pow 0       = return$^c_p$ (lift$^c_p$ (fun x -> 1))
pow (suc n) =
  let return$^c_p$ C = pow n in
  nb<<return$\color{NavyBlue}^c_p$ (lift$\color{NavyBlue}^c_p$ (fun x -> x * (unlift$\color{NavyBlue}^c_p$ (C)) x))>>nb
      \end{lstlisting}
    };
    \draw[NavyBlue,visible on=<4>] (0.09,0.465) -- (0.475,0.465);

    \node[text width=0.58\paperwidth,visible on=<5->] at (0.31,0.6) {
      \begin{lstlisting}
pow : nat -> v|$^c_p$^|$^c_p$(nat -> nat)
pow 0       = return$^c_p$ (lift$^c_p$ (fun x -> 1))
pow (suc n) =
  let return$^c_p$ C = pow n in
  return$^c_p$ (lift$^c_p$ (fun x -> x * (unlift$^c_p$ (C)) x))
      \end{lstlisting}
    };

    \node[visible on=<5->,text width=\textwidth,anchor=west] at (0.015,0.25) {\lstinline!pow 2! gives\\ \lstinline!return$^c_p$ (lift$^c_p$ (fun x$_1$ -> x$_1$ * (fun x$_2$ -> x$_2$ * (fun x$_3$ -> 1) x$_2$) x$_1$))!};
    \node[visible on=<6>] at (0.5,0.1) {But this requires {\color{red}\(2\) store-loads}!};
  \end{tikzpicture}  
\end{frame}

\begin{frame}[fragile]{Separation of Concern in Metaprogramming}
  \begin{tikzpicture}[fulloverlay]
    \node[text width=0.58\paperwidth] at (0.31,0.5) {
      \begin{lstlisting}
powHelper : nat -> ^|$^c_p$(nat -> nat)
powHelper 0       = lift$^c_p$ (fun x -> 1)
powHelper (suc n) =
  lift$^c_p$ (fun x -> x * (unlift$^c_p$ (powHelper n)) x))

pow : v|$^c_p$nat -> v|$^c_p$^|$^c_p$(nat -> nat)
pow dn =
  let return$^c_p$ n = dn in
  return$^c_p$ (powHelper n)
      \end{lstlisting}
    };
  \end{tikzpicture}  
\end{frame}

\begin{frame}[fragile]{Resource Availability Example}
  \begin{tikzpicture}[fulloverlay]
    \node[visible on=<1>,text width=0.58\paperwidth] at (0.31,0.5) {
      \begin{lstlisting}
withAlloc : nat -o nat -> (nat array -o nat * nat array) -o nat
read      : nat array -o nat        -o nat * nat array
write     : nat array -o nat -o nat -o nat array
      \end{lstlisting}
    };
    \node[visible on=<2>,text width=0.58\paperwidth] at (0.31,0.4) {
      \begin{lstlisting}
foldRight : nat ->
            v|$^{c}_{p}$^|$^{c}_{p}$ ((nat -o nat -o nat) ->
                    nat -o
                    nat array -o
                    (nat * nat array))
foldRight 0       =
  return$^{c}_{p}$ (lift$^{c}_{p}$ (fun f -> fun z -o fun xs -o (z , xs)))
foldRight (suc n) =
  let return$^{c}_{p}$ C = foldRight n in
  let return$^{c}_{p}$ N = codifyNat n in
  return$^{c}_{p}$ (lift$^{c}_{p}$ (fun f -> fun z -o fun xs0 -o
    llet (v, xs1) = read xs0 (unlift$^{c}_{p}$ N) in
    (unlift$^{c}_{p}$ C) f (f z) xs1))
      \end{lstlisting}
    };
  \end{tikzpicture}  
\end{frame}

\begin{frame}[fragile]{Mode Specification}
  \begin{tikzpicture}[fulloverlay]
    \node at (0.5,0.65) {\(\vdots\)};
    \draw (0.4,0.6) rectangle (0.5,0.5);
    \draw (0.5,0.6) rectangle (0.6,0.5);
    \draw
    (0.4,0.5) rectangle (0.6,0.4)
    node[pos=0.5] {\(\alt<2->{{\color{OliveGreen}m}}{m}\) : \(\alt<3->{{\color{BrickRed}\{\ldots\}}}{\{\ldots\}}, \alt<3->{{\color{MidnightBlue}\{\ldots\}}}{\{\ldots\}}\)};
    \draw (0.4,0.4) rectangle (0.5,0.3);
    \draw (0.5,0.4) rectangle (0.6,0.3);
    \node at (0.5,0.25) {\(\vdots\)};

    \draw[OliveGreen,visible on=<2->] (0.415,0.423) -- (0.44,0.423);
    \node[OliveGreen,visible on=<2->] (name) at (0.2,0.4) {name of a mode};
    \draw[OliveGreen,->,visible on=<2->] (name.east) -- (0.4275,0.418);

    \draw[BrickRed,visible on=<3->] (0.457,0.423) -- (0.513,0.423);
    \node[BrickRed,text width=10em,align=center,visible on=<3->] (st) at (0.2,0.3) {Structural rules \\allowed in the mode};
    \draw[BrickRed,->,visible on=<3->] (st.east) -- (0.485,0.418);

    \draw[MidnightBlue,visible on=<4->] (0.53,0.423) -- (0.58,0.423);
    \node[MidnightBlue,align=center,visible on=<4->] (ty) at (0.2,0.2) {Types allowed in the mode};
    \draw[MidnightBlue,->,visible on=<4->] (ty.east) -- (0.555,0.418);

    \draw[->,visible on=<5->] (0.63,0.4) -- (0.63,0.2);
    \node[text width=12em,align=center,visible on=<5->] at (0.8,0.3) {Modes that can use \\values of this mode};
    \draw[->,visible on=<6->] (0.63,0.5) -- (0.63,0.7);
    \node[text width=12em,align=center,visible on=<6->] at (0.8,0.6) {Modes whose values can \\be used in this mode};
  \end{tikzpicture}
\end{frame}

\begin{frame}[fragile]{Examples of Mode Specifications}
  \scriptsize
  \begin{tikzpicture}[fulloverlay]
    \draw
    (0.1,0.7) rectangle (0.3,0.625)
    node[pos=0.5] (stlc) {\(p\) : \(\{\mathbf{Co}, \mathbf{Wk}\}, \{\multimap\}\)};

    \node at ($(stlc.south)-(0,0.05)$) {STLC};

    \draw
    (0.4,0.775) rectangle (0.6,0.7)
    node[pos=0.5] {\(c\) : \(\{\mathbf{Co}, \mathbf{Wk}\}, \{\uparrow\}\)};
    \draw
    (0.4,0.7) rectangle (0.6,0.625)
    node[pos=0.5] (meta) {\(p\) : \(\{\mathbf{Co}, \mathbf{Wk}\}, \{\downarrow,\multimap\}\)};

    \node[text width=15em,align=center] at ($(meta.south)-(0,0.05)$) {Metaprogramming system \\based on S4 modal logic};

    \draw
    (0.7,0.775) rectangle (0.9,0.7)
    node[pos=0.5] {\(u\) : \(\{\mathbf{Co}, \mathbf{Wk}\}, \{\uparrow\}\)};
    \draw
    (0.7,0.7) rectangle (0.9,0.625)
    node[pos=0.5] (linear) {\(p\) : \(\{\}, \{\downarrow,\multimap\}\)};

    \node[text width=15em,align=center] at ($(linear.south)-(0,0.05)$) {Linear type system with \(!\)};

    \draw
    (0.1,0.375) rectangle (0.3,0.3)
    node[pos=0.5] {\(p\) : \(\{\mathbf{Co}, \mathbf{Wk}\}, \{\uparrow,\multimap\}\)};
    \draw
    (0.1,0.3) rectangle (0.3,0.225)
    node[pos=0.5] (privacy) {\(s\) : \(\{\mathbf{Co}, \mathbf{Wk}\}, \{\downarrow\}\)};

    \node[text width=15em,align=center] at ($(privacy.south)-(0,0.05)$) {Privacy tracking type system};

    \draw
    (0.4,0.45) rectangle (0.6,0.375)
    node[pos=0.5] {\(c\) : \(\{\mathbf{Co}, \mathbf{Wk}\}, \{\uparrow\}\)};
    \draw
    (0.4,0.375) rectangle (0.6,0.3)
    node[pos=0.5] {\(p\) : \(\{\mathbf{Co}, \mathbf{Wk}\}, \{\uparrow,\downarrow,\multimap\}\)};
    \draw
    (0.4,0.3) rectangle (0.6,0.225)
    node[pos=0.5] (metaprivacy) {\(s\) : \(\{\mathbf{Co}, \mathbf{Wk}\}, \{\downarrow\}\)};

    \node[text width=15em,align=center] at ($(metaprivacy.south)-(0,0.05)$) {System with privacy and metaprogramming};

    \draw
    (0.65,0.45) rectangle (0.8,0.375)
    node[pos=0.5] {\(c\) : \(\{\}, \{\uparrow\}\)};
    \draw
    (0.8,0.45) rectangle (0.95,0.375)
    node[pos=0.5] {\(u\) : \(\{\mathbf{Co}, \mathbf{Wk}\}, \{\uparrow\}\)};
    \draw
    (0.65,0.375) rectangle (0.95,0.3)
    node[pos=0.5] (metalinear) {\(p\) : \(\{\}, \{\downarrow,\multimap\}\)};

    \node[text width=15em,align=center] at ($(metalinear.south)-(0,0.125)$) {System with linearity and metaprogramming};

    % \node at (0.5,0.45) {\(\alt<2->{{\color{OliveGreen}m}}{m} : \alt<3->{{\color{BrickRed}\{\ldots\}}}{\{\ldots\}}, \alt<3->{{\color{MidnightBlue}\{\ldots\}}}{\{\ldots\}}\)};
    % \draw (0.4,0.4) rectangle (0.5, 0.3);
    % \draw (0.5,0.4) rectangle (0.6, 0.3);

    % \draw[OliveGreen,visible on=<2->] (0.415,0.423) -- (0.44,0.423);
    % \node[OliveGreen,visible on=<2->] (name) at (0.2,0.4) {name of a mode};
    % \draw[OliveGreen,->,visible on=<2->] (name.east) -- (0.4275,0.418);

    % \draw[BrickRed,visible on=<3->] (0.457,0.423) -- (0.513,0.423);
    % \node[BrickRed,text width=10em,align=center,visible on=<3->] (st) at (0.2,0.3) {Structural rules \\allowed in the mode};
    % \draw[BrickRed,->,visible on=<3->] (st.east) -- (0.485,0.418);

    % \draw[MidnightBlue,visible on=<4->] (0.53,0.423) -- (0.58,0.423);
    % \node[MidnightBlue,align=center,visible on=<4->] (ty) at (0.2,0.2) {Types allowed in the mode};
    % \draw[MidnightBlue,->,visible on=<4->] (ty.east) -- (0.555,0.418);

    % \draw[->,visible on=<5->] (0.63,0.4) -- (0.63,0.2);
    % \node[text width=12em,align=center,visible on=<5->] at (0.8,0.3) {Modes that can use \\values of this mode};
    % \draw[->,,visible on=<6->] (0.63,0.5) -- (0.63,0.7);
    % \node[text width=12em,align=center,visible on=<6->] at (0.8,0.6) {Modes whose values can \\be used in this mode};
  \end{tikzpicture}
\end{frame}

\begin{frame}{Current Status}
  \begin{itemize}
  \item Preservation/progress are proved for declarative typing rules
  \item Implementation based on algorithmic typing rules
  \item Full embedding of \(\lambda^{\Box}\) is proved and mechanized
  \end{itemize}
\end{frame}

\end{document}
